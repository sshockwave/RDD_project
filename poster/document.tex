% Modified based on Xiaoming Sun's template and https://www.overleaf.com/latex/templates/cs6780-assignment-template/hgjyygbykvrf

\documentclass[a4 paper,12pt]{article}
\usepackage[inner=2.2cm,outer=2.2cm,top=2.5cm,bottom=2.5cm]{geometry}
\usepackage{setspace}
\usepackage[rgb]{xcolor}
\usepackage{verbatim}
\usepackage{subcaption}
\usepackage{fancyhdr}
\usepackage[colorlinks=true, urlcolor=blue, linkcolor=blue, citecolor=blue]{hyperref}
\usepackage{booktabs}
\usepackage{amsmath,amsfonts,amsthm,amssymb}
\usepackage{setspace}
\usepackage{fancyhdr}
\usepackage{lastpage}
\usepackage{extramarks}
\usepackage{indentfirst}
\usepackage{chngpage}
\usepackage{soul,color}
\usepackage{bm}
\usepackage{graphicx,float,wrapfig}
\usepackage{tikz}
\usepackage{makecell}
\setlength{\parindent}{2em}
\usetikzlibrary{positioning}
\usetikzlibrary{arrows,graphs}
\newcommand{\homework}[3]{
   \pagestyle{myheadings}
   \thispagestyle{plain}
   \newpage
   \setcounter{page}{1}
   \noindent
   \begin{center}
   \framebox{
        \vbox{\vspace{2mm}
        \hbox to 6.28in { {\bf Causal and Statistical Inference \hfill} {\hfill {\rm #2} {\rm #3}} }
        \vspace{4mm}
        \hbox to 6.28in { {\Large \hfill #1  \hfill} }
        \vspace{3mm}}
   }
   \end{center}
   \vspace*{4mm}
}
\newcommand\numberthis{\addtocounter{equation}{1}\tag{\theequation}}

% Custom defined operators
\DeclareMathOperator{\Do}{do}

\begin{document}
\homework{Poster Report: RDD}{Kai Su,}{Qihang Chen}

\section{Analysis with Baysian Network}
\subsection{Introduction to RDD}
When the running variable $X$ takes value at different side of a threshold value $t$, there will be a treatment(marked $w=1$) having an effect on the output $Y$ on one side of $X=t$ and no treatment(marked $w=0$) on the other side. Our goal is to measure the average effect $\tau$ solely caused by the treatment. That is, $\tau=E[Y|\Do(W)=1]-E[Y|\Do(W)=0]$.

Such effect appears as a sudden increase or decrease of $y$ in the neighborhood of $X=t$(See Fig.\ref{fig:rddintro}). Therefore, $\tau$ is traditionally estimated using the see effect of $W$. Formally put,

\begin{equation}
   \hat\tau=\lim_{x\to t^+}E[Y|X=x]-\lim_{x\to t^-}E[Y|X=x].\label{eqn:est_lim}
\end{equation}

For simplicity, denote 
\begin{align}
   E[Y|X=t_+]=\lim_{x\to t^+}E[Y|X=x],\\
   E[Y|X=t_-]=\lim_{x\to t^-}E[Y|X=x].
\end{align}

However, we are lack of data near the threshold in most cases, difficult for us to calculate these values on the threshold directly. Thus we need some regression to infer the relation between $X$ and $Y$ so as to predict $E[Y|X=t_+]$ and $E[Y|X=t_-]$.

A common practice is to apply linear regression on both sides. The data near the threshold is more valuable, thus we can set a bandwidth $b$, which is the largest distance where data are taken into account; and a kernel $k$, which assigns weights to data. Denote such an estimator as $\hat\tau(b,k)$.

\begin{figure}[h]
	\centering
   \includegraphics[scale=1]{RDDintro.jpg}
   \caption{An example of dataset in RDD}
   \label{fig:rddintro}
\end{figure}

\subsection{Baysian Network}
%对应作业第一部分

Basically, running variable $X$ will decide $W$ and affect $Y$; $W$ will have effect on $Y$, as in Fig.\ref{fig:basic_rd_bns}.

\begin{figure}[h]
	\centering
   \includegraphics[scale=0.3]{Figure_1.png}
   \caption{The basic BN for RDD}
   \label{fig:basic_rd_bns}
   \includegraphics[scale=0.3]{Figure_2.png}
   \caption{BN with bandwidth}
   \label{fig:bn_with_bdw}
\end{figure}

\subsubsection*{bandwidth}
When the bandwidth is sufficiently small,
\begin{align}
   P(Y|W=0,X=t)\approx P(Y|X=t_-),\\
   P(Y|W=1,X=t)\approx P(Y|X=t_+).
\end{align}

Then
\begin{align}
   \tau&=E[Y|\Do(W=1),X=t]-E[Y|\Do(w=0),X=t]\\
   &\approx E[Y|X=t_+]-E[Y|X=t_-]=\hat\tau,
\end{align}

\noindent proving that the traditional estimation in Eqn.(\ref{eqn:est_lim}) is unbiased.

In contrast, if the bandwidth $b$ is non-negligible, this selection of data specified by $b$ will cause a backdoor path between $X$ and $Y$. (See Fig.\ref{fig:bn_with_bdw}) Then the see effect observed by regression between $X$ and $Y$ is not the true causal effect, leading to a bias.

\subsubsection*{covariates}
Sometimes there are not only running variables $X$ and $Y$, but also many other variables $Z$, called covariates, may have effect on $Y$. If they are independent with $X$, the regression still works. However in some cases, $Z$ will affect both $X$ and $Y$, creating a backdoor path between $X$ and $Y$. (See Fig.\ref{bn_with_covar})

\begin{figure}[h]
	\centering
	\includegraphics[scale=0.3]{Figure_3.png}
   \caption{BN with covariates}
   \label{bn_with_covar}
	\includegraphics[scale=0.3]{Figure_4.png}
	\caption{Eliminating $Z\rightarrow X$}
	\label{bn_with_covar2}
\end{figure}

\subsection{Obstacles}
\label{sec:obstacles}
%对应作业第二部分

Under the BN shown in Fig.\ref{fig:bn_with_bdw}, the performance of the estimator $\hat \tau(b,k)$ is restricted by two factors:

\begin{enumerate}
   \item[(a)] Large variance due to lack of data;
   \item[(b)] Bias caused by fitting non-linear $X-Y$ relation with linear regression.
\end{enumerate}

When $b\to 0$, less and less data stay within the bandwidth, so the variance in (a) becomes significant. The bias in (b) converges to $0$ under smoothness assumption on the $X-Y$ relation. The conclusion is opposite when $b\to+\infty$.

If covariates create a backdoor path as shown in Fig\ref{bn_with_covar2}, the confound effect significantly becomes a source of error in cases with large bandwidth and non-linear $X-Y$ relation.

\section{Experiment and Discussion}

The bandwidth $b$ and the kernel $k$ together decides the weights of data,
yet they are separated into two independent parameters.
The bandwidth focuses on the definition of "distance",
and thus allows the reuse of kernels across problems of similar properties but with different scales and units.

Moreover, the final probability distribution of samples supplied to the regression model is a composite outcome of the weights of data and the density of actual data.
Only then can we improve the accuracy of $\hat\tau$ by manipulating $b$ and $k$.
By letting the errors stated in section \ref{sec:obstacles} be the loss function, we get a chain of methods to determine $b$ and $k$:
\begin{enumerate}
   \item Find a bandwidth $b$ which balances between
   \begin{enumerate}
      \item the lack of data;
      \item and bias caused by non-linearity of $X-Y$ relation.
   \end{enumerate}
   \item Find a kernel $k$ which
   \begin{enumerate}
      \item helps to reduce the bias caused by bandwidth;
      \item eliminates the confound effect as much as possible.
   \end{enumerate}
\end{enumerate}

\subsection{Error-minimizing bandwidth selection}

In some articles, the method of cross validation is used for selecting an optimal bandwidth.
However, this is not always accurate,
because the samples near the threshold may perform very differently.
Differences in these properties might result in a poor bandwidth for the estimation at the threshold.

If $X-Y$ relation is non-linear,
the curvature of the function $y=y(x)$ varies.
An overestimation of the curvature leads to small bandwidth,
which does not take full advantage of the data.
An underestimation of the curvature leads to large bandwidth,
which might introduce more bias from the non-linear $X-Y$ relation.

Difference of density of data is another issue in RDD particularly.
Data near the threshold might be under subjective manipulation.
For example, students might work hard not to get below the pass line,
so the number of students who is right below $60$ is much less than the number of students with other scores.
This turbulance in the density of data, again, does not help to select a good bandwidth through cross validation.

Therefore, we propose a method of quantifying the expected error at the cutoff directly, which is an estimation of the two aspects mentioned in section \ref{sec:obstacles}:

\begin{enumerate}
   \item[(a)] Using bandwidth $b$, suppose the linear regression has the result $Y=a(X-t)+b$. We use a formula to calculate the standard derivation of $b$ to represent the expected error caused by lack of data near threshold.
   \item[(b)] First use a quadratic hypothesis function on regression to find out a curve showing approximate quadratic relationship between $X-Y$(do not use bandwidth $b$). Then project all the samples onto the curve, use linear regression with bandwidth $b$ and find the difference of value between the two ways of regression on the threshold.
\end{enumerate}

Finally, we find a bandwidth who has least sum of those two kind of errors, which is the optimal bandwidth we need. (See Fig.\ref{table:opt bandwidth})

\begin{figure}[h]
	\centering
	\includegraphics[scale=1]{Figure_5.png}
	\caption{Evaluate error of different bandwidth (on the headstart dataset)}
	\label{table:opt bandwidth}
\end{figure}

\subsection{Using kernels to reduce bias caused by bandwidth}

In this section we do not consider covariates. We generate data by different $X-Y$ relation and different distribution of samples (See table \ref{table:kernels_test}). The cutoff is at $X=59$.

\begin{table}[h]
	\centering
   \begin{tabular}{|c|c|c|c|}
      \hline
      $y=$&$0.2(x-59)^2+2$&$-0.2(x-59)^2-0.4(x-59)+2$&$0.1(x-59)^3+2$\\
      \hline
      Dense middle&case00&case01&case02\\
      \hline
      Sparse middle&case10&case11&case12\\
      \hline
      Dense cutoff&case20&case21&case22\\
      \hline
      Sparse cutoff&case30&case31&case32\\
      \hline
   \end{tabular}
   \caption{$X-Y$ relations and data distributions}
   \label{table:kernels_test}
\end{table}

We randomly generated 500 groups of data and 200 different types of kernels using bezier curves. For each kernel, we calculate the difference of $Y$ at threshold compared with $Y(t)$ by true $X-Y$ relation in those 500 groups of data, getting the average to show the performance of the kernel. Thus we get the best kernel for different bandwidth.

\begin{figure}
   \centering
   \includegraphics[scale=0.3]{case30_frame0000000.png}
   \includegraphics[scale=0.3]{case30_frame0000031.png}
   \includegraphics[scale=0.3]{case30_frame0000049.png}
   \caption{Illustrated best kernels under different bandwidths}
   \label{fig:kernels}
\end{figure}
\subsubsection*{Kernels vary with density of data}

A typical result(case30) is shown in Fig.\ref{fig:kernels}. The $x$-axis is the distance from the cutoff, and the $y$-axis is the weight of the kernel. At a fixed bandwidth, the kernel does not gather around the cutoff at first, but chooses to spread out the weight to larger distance to the cutoff when bandwidth is small.

This initial behavior of the kernel is related to the error mentioned in section \ref{sec:obstacles}(a). The lack of samples cause large uncertainty. Assigning more weight to those farther samples basically equivalents to obtaining more samples, and by doing so, the kernel is able to lower the variance of estimation. As the bandwidth $b$ becomes larger, the weights concentrate to the cutoff to lower the bias caused by the nonlinearity in $X-Y$ relation, corresponding to section \ref{sec:obstacles}(b).

\subsubsection*{Kernels vary with density of data}

Another observation is that, the distribution of samples along $X$ axis really affect the performance of kernels. The reason is the distribution of samples affect the randomness of sampling, thus we need to use the kernel to adjust the sampling method.

Comparing the result in different rows(in html files, can not be shown in pdf), we can find a regular pattern that adding weight on those samples where samples along the $X$ axis are sparse will give better performance. 

It is intuitive to explain this observation. If the samples are evenly distributed, it is easier to infer the information from data. But if some areas on $X$ axis has larger density of samples, giving less weight to them can mimic an evenly distributed set of data. Then the kernel will trend to trust samples in the sparse area.

\subsection{ Differences after adding covariates}
We generate data by different $X-Y$ relation and different ways $Z$ affect $X,Y$ (See table \ref{table:kernels_test adding covariates}) where the samples are uniformly distributed.

\begin{figure}
	\centering
	\includegraphics[scale=0.45]{case41_frame0000035.png}
	\caption*{41}
	\includegraphics[scale=0.49]{casez01_frame0000035.png}
	\caption*{z01}
	\includegraphics[scale=0.49]{casez11_frame0000035.png}
	\caption*{z11}
	\includegraphics[scale=0.45]{casez21_frame0000035.png}
	\caption*{z21}
	\caption{Illustrated best kernels under different ways of adding covariates}
	\label{fig:kernels2}
\end{figure}

After adding covariates, there will be another backdoor path caused by $Z$, and the perfomance of different kernels becomes different (See Fig.\ref{fig:kernels2} ). This tells us our selection of bandwidth should concern the effect of covariates, and it will be unbiased only when the BNS is the same as Fig.\ref{bn_with_covar2}.

\begin{table}[h]
	\centering
	\begin{tabular}{|c|c|c|c|}
		\hline
		$f(x)=$&$0.2(x-59)^2+2$&$-0.2(x-59)^2-0.4(x-59)+2$&$0.1(x-59)^3+2$\\
		\hline
		\makecell{$x=x_0$\\$y=f(x)$}&case40&case041&case42\\
		\hline
		\makecell{$x=x_0+z$\\$y=f(x)+0.3z$}&casez00&casez01&casez02\\
		\hline
		\makecell{$x=x_0+z^2$\\$y=f(x)+0.3z$}&casez10&casez11&casez12\\
		\hline
		\makecell{$x=x_0+z$\\$y=f(x)+0.5Z^2$}&casez20&casez21&casez22\\
		\hline
	\end{tabular}
	\caption{$X-Y$ relations and ways of adding $Z$}
	\label{table:kernels_test adding covariates}
\end{table}

\subsection{Using weights to eliminate bias caused by confound effects}
After adding covariates, there might be another backdoor path caused by $Z$, and the perfomance of different kernels may become different.

In order to eliminate this bias, we need to find a bandwidth, to make sure among the included samples, $Z$ is independent with $X$. (See Fig.\ref{bn_with_covar2})

\end{document} 
